\documentclass{article}

%\parskip 16pt
\topmargin 0pt
\textheight 8.5in
\textwidth 7in
\oddsidemargin -0.25in
\evensidemargin 0in
\widowpenalty=50000
\clubpenalty=50000



%\usepackage{times}
%\usepackage{pdfsync}
\usepackage{etaremune}

\pagestyle{myheadings} 
%\markright{Sanjiv Ranjan Das: \today \dotfill }

\renewcommand{\familydefault}{cmss}

\begin{document}



%\vspace*{0.5in}

\begin{center}

\today

{\bf Curriculum Vitae \\[0.4cm]
SANJIV RANJAN DAS} \\[0.6cm]

     William and Janice Terry Professor of Finance and Data Science\\
        Santa Clara University, 
        Leavey School of Business\\
        Department of Finance, 316M Lucas Hall\\
        500 El Camino Real, 
        Santa Clara, CA 95053.\\
        Email: srdas@scu.edu, Tel: 408-554-2776\\
        Web: http://srdas.github.io/

%     Current: Visiting Professor, University of California\\
%       F694, Haas School of Business\\
%       Berkeley, CA 94720-1900\\
%       510-642-3421 (O)\\
%       Email: sdas@haas.berkeley.edu\\
%       Web: www.haas.berkeley.edu/~sdas                


%     Graduate School of Business Administration \\
%     Soldiers Field, Boston, MA 02163 \\
%     617-495-6080 (O)\\
%     617-496-6592 (F)\\
%     Email: sdas@hbs.edu \\
%     Web address: www.people.hbs.edu/sdas \\

\end{center}

%\vspace*{1in}

\begin{description}
\item[EDUCATION] \mbox{}

\begin{itemize}
\setlength\itemsep{-0.1em}

\item University of California, Berkeley, M.S. in Computer Science (Theory),
December 2000. \\
Master's Thesis: ``Lattice Excursions in Financial Models''.
  
\item New York University, Ph.D in Finance, September 1994.\\
Ph.D. Dissertation: ``Interest Rate Shocks, Characterizations of the Term
Structure, and the Pricing of Interest Rate Sensitive Contingent
Claims.''

\item New York University, M.Phil in Finance, May 1992.

\item Indian Institute of Management, Ahmedabad, MBA, May 1984.

\item Indian Institute of Cost \& Works Accountants of India, AICWA, May 1983.

\item University of Bombay, B.Com in Accounting and Economics, May 1982.

\end{itemize}

\end{description}

\begin{description} 
\item[HONORS \& AWARDS, APPOINTMENTS] \mbox{}

\begin{etaremune}
\setlength\itemsep{-0.1em}

\item Best paper award at the R Finance conference, (Chicago 2016), for the paper ``An Index-Based Measure of Liquidity.''

\item First Prize in the MIT Center for Financial Policy Contest to define SIFIs (2016).

\item Best paper award at the R/Finance conference (Chicago, May 2015) for ``Matrix Metrics: Network-Based Systemic Risk Scoring.'' 

\item 2014 GARP (Global Association of Risk Professionals) Risk Management Research Program Award (with Seoyoung Kim).

\item National Stock Exchange Best Paper Award, at the Center for Analytic Finance Conference, ISB, Hyderabad, for the paper titled ``Credit Spreads with Dynamic Debt,'' 2013. (with Seoyoung Kim)

\item Consortium for Systemic Risk Analytics, MIT, Academic Board Member, 2012--.

\item Santa Clara University Award for Sustained Excellence in Scholarship, 2012. 

\item Leavey School Research Award (2012) for the paper with George Chacko that won the SPIVA Award. 

\item Dauphine-Amundi Award for Asset Management Proposal on Liability-Directed Investing (2012). (with Seoyoung Kim and Meir Statman.)

\item Expert Panel, Center for Financial Research and Planning (CAFRAL), Reserve Bank of India, 2012--. 

\item Second Prize, SPIVA Research Award 2011-12, from McGraw-Hill and Standard \& Poors. (with George Chacko) 

\item CFA Institute, Capital Markets Policy Council (CPMC), 2011-- .

\item Program coordinator, Risk Management, Risk Measurement and Derivatives, at the Federal Deposit Insurance Corporation, Center for Financial Research (FDIC-CFR), 2008 - .

\item Santa Clara University-wide Award for Recent Achievement in Scholarship (2007). 

\item Nominated for the Smith-Breeden Prize for the best paper in the {\it Journal of Finance} in 2005, for the paper ``Systemic Risk and International Portfolio Choice'' (with Raman Uppal). 

\item {\it Breetwor Faculty Fellow}, Santa Clara University, 2002-2004,
2005-2007.

\item Western Finance Association Caesarea Center Award 
for the best paper on risk management at the WFA meetings
(Portland, Oregon) for the paper ``Common Failings:
How Corporate Defaults are Correlated'', 2005.  (with Darrell
Duffie and Nikunj Kapadia).

\item FDIC research fellowship for the project 
``Conditionally Correlated Default'' (with Darrell
Duffie and Nikunj Kapadia), 2004. 

\item Extra-Ordinary research award, Santa Clara University: 2004-2005.

\item Extra-Ordinary faculty award, Santa Clara University: 2000-2001, 
2001-2002, 2002-2003, 2003-2004, 2006-2007.

\item Award for the best paper presented at the 2002 meetings
of INQUIRE for the paper ``Systemic Risk and International Portfolio
Choice'' (with Raman Uppal). 

\item Second best-paper award from the {\it Journal of Banking and
Finance} for the article ``A Theory of Banking Structure'' (co-authored
with Ashish Nanda), 2001.

\item {\it Dean Witter Foundation Faculty Fellowship}, Santa Clara University,
2000-2002.

\item {\it Price Waterhouse Coopers} Risk Institute Research Sponsorship
Award, 2000.

\item {\it Association for Investment Management and Research (AIMR)
 Research Foundation Advisory Board}, 2000-2002.

\item {\it Faculty Research Fellow}, National Bureau of Economic
Research (NBER), 1996 to 1999.

\item Deloitte and Touche {\it Chair in Risk Management} at the
University of Antwerp, Belgium, 1998.

\item Merrill Lynch {\it Faculty Sponsorship Award}, Harvard Business
School, 1997.

\item {\it Best Thesis Award} from the Financial Management Association,
sponsored by the American Institute
of Individual Investors, 1995. 

\item {\it Best Thesis Award}, New York University, 1995 for the Ph.D.
dissertation in the graduating class. 

\item Canadian Investment Review {\it Academic Sponsorship Award} for
the proposal ``How Diversified are Internationally Diversified
Portfolios'' (with Prof. Raman Uppal, University of British Columbia), 1995. 

\item {\it Best Paper Award} for the paper ``Jump Diffusion Processes
and the Bond Markets'', awarded by the Financial Management Association
of America, and sponsored by the Fixed-Income Analysis Society of New
York, at the meetings in St. Louis, 1994. 

\item Beta Gamma Sigma, New York, 1994. 

\item C.W. Nichols Fellowship, New York University, 1993-94. 

\item Jules I. Bogen Fellowship, New York University, 1992-93. 

\item Doctoral Fellowship in Finance, New York University, 1990-92. 

\item Citicorp Chairman's Award for Excellence, 1985. 

\item {\it First Rank} in the MBA Class (of 165 students),
Indian Institute of Management, 1984. 

\item Industry Fellowship, Indian Institute of Management, Ahmedabad, 1983-84. 

\item {\it Gold Medalist} in Financial Management at the
Accountancy Examinations of the Institute of
Cost \& Works Accountants of India, 1983. 

\item {\it First Rank and Gold Medalist} at the University of Bombay B.Com 
Examinations (of 13,563 candidates), 1982. 

\item National Talent Search Fellowship (awarded annually to 150
students based on an All-India Examination), 1982. 

\end{etaremune}

\end{description}



\begin{description} 
\item[ACADEMIC EXPERIENCE] \mbox{}

\begin{itemize}
\setlength\itemsep{-0.1em}

\item Santa Clara University\\
\hspace*{0.25in}William and Janice Terry Professor of Finance and Data Science, September 2012 - date.\\
\hspace*{0.25in}Chair, Dept. of Finance, July 2007 - December 2011.\\
\hspace*{0.25in}Professor, July 2006 - date.\\
\hspace*{0.25in}Director of Faculty Research, July 2006 - 2007.\\
\hspace*{0.25in}Associate Professor, July 2000 - June 2006.
%\hspace*{0.25in}{Courses:}\\
%\hspace*{0.5in} Money and Capital Markets, 2001-2006 (undergraduate)\\
%\hspace*{0.5in} Financial Engineering, 2001-2007, 2010-2013 (MBA)\\
%\hspace*{0.5in} Investments, 2002-2005 (MBA)\\
%\hspace*{0.5in} Introduction to Options, 2003 (MBA)\\
%\hspace*{0.5in} Introduction to Default Models 2003-2006 (MBA)\\
%\hspace*{0.5in} Monte Carlo Simulation in Finance, 2003-2006 (MBA)\\
%\hspace*{0.5in} Introduction to Stochastic Models 2004-2005 (MBA)\\
%\hspace*{0.5in} Quantitative Business Models 2007, 2012-2013 (MBA)\\
%\hspace*{0.5in} Creating Value 2006-2009 (Exec MBA)\\
%\hspace*{0.5in} Financial Management 2006--2012 (Exec MBA) \\
%\hspace*{0.5in} Mathematical Finance 2008-2009, 2011-2013 (MBA)\\
%\hspace*{0.5in} Quantitative Methods for Finance 2010 (undergraduate)\\

\item Indian School of Business, Hyderabad, India. \\
\hspace*{0.25in} Short-term Visiting Professor since 2006. 
%\hspace*{0.5in} Financial Engineering, Nov-Dec 2006, 2008--2010, 2012 \\

\item EDHEC Business School, Singapore.\\
\hspace*{0.25in} Short-term Visiting Professor, 2012, 2014. 

\item University of California, Berkeley.\\
\hspace*{0.25in}Visiting Associate Professor, July 1999 - June 2000. 
%\hspace*{0.5in} Corporate Finance, 1999 (second year MBA elective)\\
%\hspace*{0.5in} Corporate Finance, 2000 (second year MBA elective).


\item Harvard University, Graduate School of Business Administration.\\
\hspace*{0.25in}Associate Professor, July 1999 - June 2000. \\
\hspace*{0.25in}Assistant Professor, Finance Area, 1994 - 1999.
%\hspace*{0.5in} First Year Finance, 1995 (1st year MBA).\\
%\hspace*{0.5in} Capital Markets, 1996, 1997, 1998 and 1999 \\
%\hspace*{0.5in} (second year MBA elective). \\
%\hspace*{0.5in} Finance Theory in Continuous-Time, 1998, 1999 (PhD level).\\
%\hspace*{0.5in} Executive Program, 1998, 1999.


%\item New York University, Leonard N. Stern School of Business.\\
%\hspace*{0.25in}Teaching Assistant (part time), Finance Area.\\
%\hspace*{0.5in} Debt Markets and Instruments, 1993.  

\end{itemize}

\end{description}

\begin{description} 
\item[OTHER ACADEMIC EXPERIENCE] \mbox{}

\begin{itemize}
\setlength\itemsep{-0.1em}

\item Advisory Board, Northwestern University, Department of Industrial Engineering and Management Science (IEMS), 2016. 

\item SCU Accelerator Planning Committee (2015-16). 

\item SCU Founder Business Analytics Program (2015-16). 

\item SCU Business School Grievance Committee (2014-2016). 

\item SCU Co-chair, Dean Search Committee for the School of Business, 2014-2015.

\item SCU Mid-Probationary Review Committee, 2014.

\item SCU India Advisory Committee, 2013-2014.

\item SCU Scholarship Standards Review Committee, 2013-2014.


\item University-wide Rank and Tenure Committee\\
        Santa Clara University, 2010-2012, Chair in 2012-2013.

\item Member, Dean Search Committee\\
	Santa Clara University, 2008-2009.

\item Chairman, Department of Finance\\
	Santa Clara University, 2007-2011.

\item Director for Faculty Development\\
	Santa Clara University, 2006-2007.

\item Business School Rank and Tenure Committee \\
	Santa Clara University, 2003-04, 2004-05, 2007-08.

\item PhD Committee and PhD Finance Program Coordinator\\
        Harvard Business School, 1997-98, 1998-99.
        
\end{itemize}

\end{description}


\begin{description} 
\item[NON-ACADEMIC EXPERIENCE] \mbox{}

\item Citibank, N.A., Asia-Pacific Region, Vice-President, 1984-1990.\\
Retail banking. Trading Room Operations. Analytics.
%Managed the software team
%building trading models for the bank in the Asia-Pacific
%region. 

\end{description}


\begin{description}
\item[BOOKS] \mbox{}
\end{description}

\begin{etaremune}
\setlength\itemsep{-0.1em}

\item ``Data Science: Theories, Models, Algorithms, and Analytics'' (web book in progress)

\item ``Derivatives: Principles and Practice (2010), Rangarajan K. Sundaram \& Sanjiv R. Das, McGraw-Hill. 

\end{etaremune}


\begin{description}
\item[JOURNAL PUBLICATIONS] \mbox{}
\end{description}
%item Note: (*) denotes work done wholly or partly while at SCU.

\begin{etaremune}
\setlength\itemsep{-0.1em}

\item ``An Index-Based Measure of Liquidity,'' (with George Chacko and Rong Fan), (2016), {\it Journal of Banking and Finance}, v68, 162--178. [This paper won the S\&P SPIVA 2012 Award for innovation of an index. It also won the best paper award at the R Finance conference (Chicago 2016).]


\item ``Matrix Metrics: Network-Based Systemic Risk Scoring.'' (2016). The {\it Journal of Alternative Investments}, Special Issue on Systemic Risk, v18(4), 33--51. [This paper won the First Prize in the MIT-CFP competition for the best paper on SIFIs (systemically important financial institutions).][Best paper award at R/Finance 2015.]


\item ``Credit Spreads with Dynamic Debt,'' (with Seoyoung Kim), (2015), {\it Journal of Banking and Finance} 50, 121--140. [National Stock Exchange Best paper award at the ISB Center for Analytical Finance Conference, 2013.]

\item ``Text and Context: Language Analytics for Finance,'' (2014), {\it Foundations and Trends in Finance}, v8(3), 145--260. 


\item ``Did CDS Trading Improve the Market for Corporate Bonds,'' (with Madhu Kalimipalli and Subhankar Nayak), (2014), {\it Journal of Financial Economics} 111, 495--525.

\item ``Strategic Loan Modification: An Options-Based Response to Strategic Default,'' (with Ray Meadows), (2013), {\it Journal of Banking and Finance} 37, 636--647.

\item ``Options and Structured Products in Behavioral Portfolios,'' (with Meir Statman), (2013), {\it Journal of Economic Dynamics and Control}, v37(1), 137-153. 

\item ``The Principal Principle'', (2012), {\it Journal of Financial and Quantitative Analysis} 47(6), 1215--1246.  

\item ``Extracting, Linking and Integrating Data from Public Sources: A Financial Case Study,'' (with Douglas Burdick, Mauricio A. Hern�ndez, Howard Ho, Georgia Koutrika, Rajasekar Krishnamurthy, Lucian Popa, Ioana Stanoi, Shivakumar Vaithyanathan), (2011), {\it IEEE Data Engineering Bulletin} 34(3), 60-67.

\item ``Polishing Diamonds in the Rough: The Sources of Syndicated Venture Performance'' (with Hoje Jo and Yongtae Kim), (2011), {\it Journal of Financial Intermediation} 20(2), 199--230. 

\item ``Portfolio Optimization with Mental Accounts'' (with Harry Markowitz, Jonathan Scheid and Meir Statman), (2010), {\it Journal of Financial and Quantitative Analysis} 45(2), 311--334.

\item ``The Long and Short of It: Why are Stocks with Shorter Run Lengths Preferred?'' (2010) (with Priya Raghubir), {\it Journal of Consumer Research}, 36(6), 964-982.  


\item ``Run Lengths and Liquidity'' (with Paul Hanouna), (2010), {\it Annals of Operations Research}, 176(1), 127-152. 

\item ``Implied Recovery,'' (2009), (with Paul Hanouna), {\it Journal of Economic Dynamics and Control}, 33(11), 1837-1857. 

\item ``Accounting-based versus market-based cross-sectional models of CDS spreads'' (2009), (with Atulya Sarin and Paul Hanouna), {\it Journal of Banking and Finance}, 33, 719-730. 

\item ``Hedging Credit: Equity Liquidity Matters'' (2009). 
(with Paul Hanouna), {\it Journal of Financial Intermediation}, v18(1), 112-123.  

\item ``An Integrated Model for Hybrid Securities,'' (2007).
(with Rangarajan Sundaram), {\it Management Science}, v53, 1439-1451.

\item ``Yahoo for Amazon: Opinion Extraction from Small Talk on the Web,'' (2007). 
(with Mike Chen), {\it Management Science}, v53, 1375-1388.    

\item ``Common Failings: How Corporate Defaults are Correlated'', (2007), 
(with Darrell Duffie, Nikunj Kapadia and Leandro Saita),
{\it Journal of Finance}, v62, 93-117. [Western Finance Association Caesarea Center Award for the best paper on risk management at the WFA meetings (Portland, Oregon, 2005).]

Reprinted in ``The Foundations of Credit Risk Analysis'', 2008, {\it The International Library of Critical Writings in Economics 211}, Eds: Willi Semmler and Lucas Bernard, Edward Elgar, UK.


\item ``A Clinical Study of Investor Discussion and
Sentiment,'' (2005), (with Asis
Martinez-Jerez and Peter Tufano, HBS), 
{\it Financial Management}, v34(5), 103-137.


\item ``International Portfolio Choice with Systemic Risk,'' 
(with Raman Uppal, London Business School), (2004), {\it Journal
of Finance}, v59(6), 2809-2834. [Nominated for the Smith-Breeden Prize for the best paper in the {\it Journal of Finance} in 2005.][Award for the best paper presented at the 2002 meetings of INQUIRE.]


\item ``Fee Speech: Signaling, Risk-sharing and the Impact of Fee
Structures on Investor Welfare,'' 2002, (with Rangarajan Sundaram),
{\it The Review of Financial Studies}, v15(5), 1465-1497.


\item ``A Discrete-Time Approach to No-Arbitrage Pricing of Credit
Derivatives with Rating Transitions,'' (with Viral Acharya and
Rangarajan Sundaram), 2002, {\it Financial Analysts
Journal}, May-June, 28-44.



\item ``The Surprise Element: Jumps in Interest Rates,''
%previously titled ``Poisson-Gaussian Processes and the Bond Markets,''
%NBER Working Paper No. 6631, 
2002, {\it The Journal of Econometrics}, v106, 27-65.

\item ``Pricing Interest Rate Derivatives: A General Approach,''
(with George Chacko), 2002, {\it The Review of Financial 
Studies}, v15(1), 195-241.





\item
``A Discrete-Time Approach to Arbitrage-Free Pricing of Credit
Derivatives,'' 2000 (with Rangarajan Sundaram), 
{\it Management Science}, v46(1), 46-62.  


 
\item
``A Case for Theory Driven Experimental Enquiry,'' 1999, (with Priya
Raghubir), {\it Financial Analysts Journal}, Nov-Dec, v55(6), 56-79.

\item
``A Direct Discrete-Time Approach to
Poisson-Gaussian Bond Option Pricing in the Heath-Jarrow-Morton 
Model,'' 1999, {\it Journal of Economic Dynamics and Control}, v23(3), 333-369.
 
\item
``A Theory of Optimal Timing and Selectivity,'' 
(with George Chacko), 1999, {\it Journal of
Economic Dynamics and Control}, v23(7), 929-966.

\item
``A Theory of Banking Structure,'' 1999, (with Ashish Nanda),
{\it Journal of Banking and Finance}, v23(6), 863-895. 
%[This paper was voted the second-best paper of the year at the journal.]

\item
``Of Smiles and Smirks: A Term Structure Perspective,'' 1999, (with
Rangarajan Sundaram), {\it Journal of Financial and Quantitative
Analysis}, v34(2), 211-240.


\item ``The Central Tendency: A Second Factor in Bond Yields,'' 1998,
(with Silverio Foresi and Pierluigi Balduzzi), {\it The Review of
Economics and Statistics}, v80(1), 60-72.

\item
``Efficiency with Costly Information: A Reinterpretation of
Evidence from Managed Portfolios,'' (with Edwin Elton, Martin Gruber and Matt 
Hlavka), {\it Review of Financial Studies}, vol. 6(1), 1993, pp 1-22. 

Presented and Reprinted in the Proceedings of The Seminar on the
Analysis of Security Prices at the Center for Research in Security
Prices at the University of Chicago, Graduate School of Business.

%\end{etaremune}

\begin{description}
\item[MORE JOURNAL PUBLICATIONS] \mbox{}
\end{description}

%\begin{etaremune}

\item ``Coming up Short: Managing Underfunded Portfolios in
a LDI-ES Framework,'' (2014), (with Seoyoung Kim and Meir Statman), {\it Journal of Portfolio Management}, v41(1), 95-108. 

\item ``Going for Broke: Restructuring Distressed Debt Portfolios,'' (2014), (with Seoyoung Kim),  {\it Journal of Fixed Income}, v24(1), 5-27. 

\item ``Digital Portfolios'', (2013), {\it Journal of Portfolio Management}, v39(2), 41--48. 

\item ``Options on Portfolios with Higher-Order Moments,'' (2009), (with Rishabh Bhandari), {\it Finance Research Letters}, v6, 122-129.


\item ``Dealing with Dimension: Option Pricing on Factor Trees'' (2009), (with Brian Granger), 
{\it Journal of Investment Management}, 7(2), 73-85. 

\item ``Correlated Default Modeling with a Forest of Binomial Trees,'' (2007),
(with Santhosh Bandreddi and Rong Fan), {\it Journal of Fixed Income}, Winter, 1-20. 

\item ``Basel II: Correlation Related Issues'', (2007) {\it Journal of Financial Services Research}, v32, 17-38.

\item ``Correlated Default Risk,'' (2006), (with Laurence
Freed, Nikunj Kapadia, and Gary Geng), {\it Journal of Fixed Income},
Fall 2006, 7-32. 


\item ``A Simple Approach for Pricing Equity 
Options with Markov Switching State Variables,'' (2006), (with
Donald Aingworth and Rajeev Motwani, Stanford University), 
{\it Quantitative Finance}, v6(2), 95-105. 

\item ``The Firm's Management of Social Interactions,'' (2005) (with Godes, D., D. Mayzlin, Y. Chen, C. Dellarocas, B. Pfeieffer, B. Libai, S. Sen, M. Shi, and P. Verlegh), {\it Marketing Letters}, v16, 415-428.�

\item ``Financial Communities,'' (2005), (with Jacob Sisk), 
{\it Journal of Portfolio Management}, v31(4), Summer, 112-123.

\item ``Markov Chain Monte Carlo methods for 
Derivative Pricing and Risk Assessment,'' 2005,
(with Alistair Sinclair, UC Berkeley), {\it Journal 
of Investment Management}, v3(1), 29-44.

\item ``Correlated Default Processes: A Criterion-Based Copula Approach,''
(with Gary Geng), (2004) {\it Journal of Investment Management}, v2(2), 44-70, 
(Special Issue on Default Risk).

Reprinted in ``Credit Risk: Models, Derivatives and Management'' (2008), editor Niklas Wagner, Chapman and Hall/CRC Financial Mathematics Series, 347-376.

\item ``The Private Equity Discount: An Empirical Examination of the
Exit of Venture Backed Companies,'' (with Murali Jagannathan,
SUNY-Binghampton, and Atulya Sarin, Santa Clara University), (2003),
{\it Journal of Investment Management}, v1(1), 152-177.


\item ``A Numerical Algorithm for Consumption/Investment Problems,''
(with Rangarajan Sundaram, New York University), (2002), 
{\it International Journal of Intelligent Systems
in Accounting, Finance and Management}, Special Issue on
Computational Methods in Economics and Finance, December, 55-69.


\item ``Bayesian Migration in Credit Ratings based on Probabilities
of Default,''  (2002), (with Rong Fan and Gary Geng), {\it
Journal of Fixed Income}, December, v12(3), 17-23.

\item ``The Impact of Correlated Default on Credit Portfolios,''
(with Gifford Fong and Gary Geng), 2001, {\it The Journal 
of Fixed Income}, 9-19.

\item
``How Diversified are Internationally Diversified Portfolios:
Time-Variation in the Covariances between International Returns,''
1998, (with Raman Uppal), {\it Canadian Investment Review}, Spring, 7-11.

\item
``Discrete-Time Bond and Option Pricing for Jump-Diffusion
Processes,'' 1997, {\it Review of Derivatives Research}, v1(3), 211-244.  

\item
``Macroeconomic Implications of Search Theory for the Labor Market,''
1997, {\it Applied Economics Letters}, December, v4, 719-723.
 

\item
``Exact Solutions for Bond and Options Prices
with Systematic Jump Risk,'' 1996, (with Silverio Foresi),
{\it Review of Derivatives Research}, v1(1), 7-24.  

\item
``A Simple Approach to Three Factor Affine Models of the
Term Structure,'' (with Pierluigi Balduzzi, Silverio Foresi and Rangarajan
Sundaram), 1996, {\it Journal of Fixed Income}, v6(3), 43-53.

\item
``Analytical Approximations of  the Term Structure
for Jump-diffusion Processes: A Numerical Analysis,'' 1996, 
(with Jamil Baz), {\it Journal of Fixed Income}, v6(1), 78-86.  

\item
``Pricing Credit Sensitive Debt when Interest Rates, Credit Ratings
and Credit Spreads are Stochastic,'' 1996, 
(with Peter Tufano), {\it The Journal of Financial Engineering},
v5(2), 161-198.

\item
``Revisiting Markov Chain Term Structure Models: Extensions and
Applications,'' 1996, {\it Financial Practice and Education}, v6(1), 33-45.  

\item
``Auction Theory: A Summary with Applications and Evidence
from the Treasury Markets,'' 1996, (with Rangarajan Sundaram),
{\it Financial Markets, Institutions and Instruments}, v5(5), 1-36.

\item
``Credit Risk Derivatives,'' {\it Journal of Derivatives}, 1995, pg 7-21. 

%\end{etaremune}


\begin{description}
\item[NON-REFEREED SHORTER ARTICLES and BOOK CHAPTERS] \mbox{}
\end{description}

%\begin{etaremune}

\item ``Big Data's Big Muscle,'' September 2016, {\it Finance and Development (IMF)}, 26--28. 

\item ``Portfolios for Investors Who Want to Reach Their Goals While Staying on the Mean-Variance Efficient Frontier,'' 2011, (with Harry Markowitz, Jonathan Scheid, and Meir Statman), {\it Journal of Wealth Management}, Fall, 14(2), 25--31.

\item ``News Analytics: Framework, Techniques and Metrics,'' The Handbook of News Analytics in Finance, May 2011, John Wiley \& Sons, U.K. 

\item ``Random Lattices for Option Pricing Problems in Finance,'' 2011, {\it Journal of Investment Management} 9(2), 88--106. 

\item ``Implementing Option Pricing Models using Python and Cython.''
(with Brian Granger), 2010, {\it Journal of Investment Management} 9(4), 73--84. 


\item ``The Finance Web: Internet Information and markets,'' 2010, {\it IEEE Intelligent Systems} 25(2), Mar/Apr, 74--78. 

\item ``Financial Applications with Parallel R,'' 2009, (with Brian Granger), {\it Journal of Investment Management}, v7(4), 66-77. 	

\item ``Recovery Rates,'' 2009, (with Paul Hanouna),
{\it Encyclopedia of Quantitative Finance}, John Wiley and Sons, U.K., pp 1505--1507.  

\item ``Recovery Swaps,'' 2009, (with Paul Hanouna), 
{\it Encyclopedia of Quantitative Finance}, John Wiley and Sons, U.K., pp 1507-1509.  

\item ``A Simple Model for Pricing Securities with a Debt-Equity Linkage,'' in 
{\it Innovations in Investment Management}, Bloomberg Press, 85-112.

%\item ``Asset Management Risk Metrics for Complex Funds,'' 
%{\it Journal of Investment Management}, 2007, forthcoming.

\item ``Credit Default Swaps,'' (with Paul Hanouna),  
{\it Journal of Investment Management}, 2006, v4(3), 93-105. 

\item ``Multiple-Core Processors for Finance Applications,'' 
{\it Journal of Investment Management}, 2006, v4(2), 76-81.

\item ``Power Laws,'' (with Jacob Sisk), 
{\it Journal of Investment Management}, 2005, v3(3), 84-91. 

\item ``Genetic Algorithms,'' 
{\it Journal of Investment Management}, 2005, v3(2), 77-82.

\item ``Recovery Risk,''
{\it Journal of Investment Management}, 2005, v3(1), 113-120. 

\item ``Venture Capital Syndication,'' (with Hoje Jo and Yongtae Kim), 
{\it Journal of Investment Management}, 2004, v2(4), 132-143. 

\item ``Modern Pricing of Interest Rate Derivatives - Book Review,''
{\it Journal of Economic Literature}, 2004, vXLII, 528-529.

\item ``Technical Analysis,'' (with David Tien),
{\it Journal of Investment Management}, 2004, v2(1), 79-85.

\item ``Liquidity and the Bond Markets'', (with Jan
Ericsson and Madhu Kalimipalli), {\it Journal of
Investment Management}, 2003, v1(4), 95-103.

\item ``Contagion'', {\it Journal of
Investment Management}, 2003, v1(3), 78-84.

\item ``Hedge Funds'', 2003, {\it Journal of
Investment Management}, v1(2), 76-81. Reprinted in
``Working Papers on Hedge Funds,'' in 
{\it The World of Hedge Funds: Characteristics and Analysis}, 
2005, World Scientific (republication of earlier paper in special
collection). 

\item ``The Internet and Investors'', 2003, {\it Journal 
of Investment Management}, v1(1), 213-217.

\item ``The Regulation of Fee Structures in Mutual Funds: A
Theoretical Analysis,'' (with Rangarajan Sundaram), 2002, 
%NBER Working Paper No 6639, 
The Courant Institute of Mathematical
Sciences, special volume on {\it Quantitative Analysis in Financial
Markets}, Volume 3. 

\item 
``A Discrete-Time Approach to Arbitrage-Free Pricing of Credit
Derivatives,'' (with Rangarajan Sundaram), 2002,
%NBER Working Paper No 6639, 
reprint in The Courant Institute of Mathematical
Sciences, special volume on {\it Quantitative Analysis in Financial
Markets}, Volume 3. 

\item ``Simply Credit: Useful things to know about Correlated
Default Risk,'' (with Gifford Fong, Laurence Freed, Gary Geng, and
Nikunj Kapadia), 2001, {\it Extra Credit}, November-December, 14-23.

\item
``Stochastic Mean Models of the Term Structure,''
(with Pierluigi Balduzzi, Silverio Foresi and Rangarajan Sundaram), 
2000, {\it Advanced Fixed-Income Valuation Tools}
edited by N. Jegadeesh and B. Tuckman,
John Wiley \& Sons, Inc., 128-161.

\item
``Interest Rate Modeling with Jump-Diffusion Processes,'' 2000,
{\it Advanced Fixed-Income Valuation Tools}, edited by
N. Jegadeesh and B. Tuckman, John Wiley \& Sons, Inc., 162-189.

\item
Comments on 'Pricing Excess-of-Loss Reinsurance Contracts against
Catastrophic Loss,' by J. David Cummins, C. Lewis, and Richard Phillips,
in {\it The Financing of Property and Casualty Risk}, Kenneth A
Froot (Ed.), University of Chicago Press, 1999, 141-145. 

\item ``Pricing Credit Derivatives,'' 1999, {\it Handbook of Credit
Derivatives}, eds J. Francis, J. Frost and J.G. Whittaker, 101-138.



\item
``On the Recursive Implementation of Term Structure Models,'' 
1998, {\it Pecunia}, The Netherlands, Summer 1998, 45-49.


\end{etaremune}

%\begin{description}
%\item[3(b). ARTISTIC PERFORMANCES]
%\item None.
%\end{description}




\begin{description}
\item[WORKING PAPERS] \mbox{}
\end{description}

\begin{etaremune}
\setlength\itemsep{-0.3em}

%\item ``Using Email to Predict Banking Failure.'' (with Seoyoung Kim, Bhushan Kothari). 

%\item ``Dynamic Systemic Risk Networks.'' (with Dan Ostrov).

%\item ``Nowcasting Bank Charge-Offs using LSTMs.'' (with IBM).

%``Systemic Risk Networks: A Large-Scale Empirical Examination.'' (with Madhu Kalimipalli and Subhankar Nayak). 

%``Deep Learning: A Survey.'' (with Subir Varma). 

\item ``Efficient Rebalancing of Taxable Portfolios.'' (with Dan Ostrov, Dennis Ding, Vincent Newell). 

\item ``Managing Rollover Risk with Capital Structure Covenants in Structured Finance Vehicles'', (with Seoyoung Kim).

%\item ``Liability Directed Investing in a Behavioral Portfolio Theory Framework,'' (with Seoyoung Kim and Meir Statman). 


\item ``The Fast and the Curious: VC Drift,'' (with Amit Bubna and Paul Hanouna). 


\item ``Venture Capital Communities,'' (with Amit Bubna and Nagpurnanand Prabhala). 

\item ``The Design and Risk Management of Structured Finance Vehicles,'' (with Seoyoung Kim).





%\Item ``Classification of Financial Information: Feature Selection
%and Categorization of Stock Message Board Postings,'' (1999),(with
%Mike Chen, Danny Tom and Jason Waddle, UC Berkeley, Computer
%Science), for submission to {\it Intelligent Systems in Finance,
%Accounting and Management}.

%\item ``Perceptual Biases in Processing Graphical Information,'' (with
%Priya Raghubir, UC Berkeley), preliminary work-in-progress.

%\item ``A Graph $\neq$ a Thousand Numbers: Biases in Processing
%Pictorial versus Digital Information'' (with Priya Raghubir, UC
%Berkeley), preliminary work-in-progress..

%\item ``Average Interest,'' (1997), (with George Chacko), NBER working 
%paper 6045. 


%\Item ``Quasi-Analytic Importance Sampling over Probability Distributions on 
%Trees with Applications for Pricing Financial Options,'' (with
%Alistair Sinclair, UC Berkeley, Computer Science Division). 

%\item ``Extracting Probabilities of Default from Debt and Equity
%Markets,'' (2002), (with Rangarajan Sundaram and Suresh Sundaresan),
%unpublished manuscript.

\end{etaremune}



\begin{description}
\item[HARVARD BUSINESS SCHOOL CASES \& TEACHING NOTES] \mbox{}
\end{description}

\begin{etaremune}
\setlength\itemsep{-0.3em}

\item ``National Insurance Corporation,'' Ref No  296-036.  (with Nils
Haugestad). Teaching Note Ref No 296-036, (with Stephen Lynagh). 
 
\item ``It's Risk, Not Return,'' Ref No  296-043.

\item ``Apex Investment Partners (B): May 1995,'' 
Ref No  296-029. (with Joshua Lerner)

\item ``Pricing Interest Rate Derivatives,'' Ref No 296-085.
(with Wai Lee)

\item ``The Banana Republic (A),'' Ref No  297-008. (with Stephen
Lynagh).  Teaching Note Ref No 298-025, (with Stephen Lynagh).

\item ``Exotic Options,'' Ref No 297-031.
(with Stephen Lynagh)

\item ``Value at Risk,'' Ref No 297-069.
(with Stephen Lynagh)

\item ``An Overview of Credit Derivatives,'' Ref No 297-086.
(with Stephen Lynagh)

\item ``The New York Stock Exchange,'' Ref No 297-107.
(with Stephen Lynagh)

\item ``Enron Corp : Credit Sensitive Notes,'' Ref No 297-099.
(with Stephen Lynagh). Teaching Note Ref No 298-100 (with Stephen Lynagh).

\item ``Treasury Inflation Protected Securities,'' Ref No 298-017. (with
Jeffrey Slovin). Teaching Note Ref No 298-017 (with Stephen Lynagh).  

\item ``Citibank Hong Kong: Capital Arbitrage in the Emerging Markets.''
Ref No 298-029. Teaching Note Ref No 298-030.

\item ``Worldwide Paper Products.'' Ref No 298-060. 
Teaching Note Ref No 298-063 (with Zerrick Bynum, Emily Keeton,
Tsutomu Noda, and Rod Parsley)
 
\end{etaremune}



\begin{description} 
\item[PRESENTATIONS AND CONFERENCE ACCEPTANCES] \mbox{}
\end{description}

\begin{etaremune}
\setlength\itemsep{-0.4em}

{\small
%item Note:  denotes the paper is subsequently published.

%\item Lehigh University (Bethlehem, PA, April 2016). ``Systemic Risk Networks.''

\item MIT GCFP Conference on Systemic Risk (Boston, September 2016). ``Dynamic Risk Networks.''

\item UseR2016 conference (Stanford, June 2016). ``More than Words: Text Analytics."

\item R Meetup (San Francisco, June 2016). ``A Brief History of the Short History of Text Analytics in Finance.''

\item R Finance conference (Chicago, May 2016). ``An Index-Based Measure of Liquidity.''

\item CapitalOne Labs (San Francisco, April 2016), ``Networks and Risk Analytics.''

\item Intel (Santa Clara, March 2016), ``The Landscape of Data Science: Algorithms and Analytics.''

\item BlackRock (San Francisco, February 2016), ``Networks and Risk Analytics.''

\item SIlicon Valley Long Term Investors (SLVTI) Group (Mountain View, February 2016). ``Goal-Based Investing with Mental Accounts.''


\item INFORMS conference (Philadelphia, November 2015). ``Systemic Risk Networks.''

\item University of Washington (Seattle, October 2015). ``Systemic Risk Networks.''

\item University of Washington (Seattle, October 2015). ``Text Analytics in Finance.''

\item San Jose Mayor's Office (San Jose, October 2015). ``Using R in Data Analytics.''

\item CDAR (Center for Data Analytics and Risk, Berkeley, October 2015). ``Systemic Risk Networks.''

\item Franklin Templeton (San Mateo, August 2015). ``The Evolving Investment Landscape.''

\item Loring Ward (San Jose, August 2015). ``Portfolio Construction with Mental Accounts.''

\item CFTC Webinar (Washington DC, July 2015). ``Systemic Risk Networks.''

\item R Meetup (Santa Clara, June 2015). ``Systemic Risk Networks.''

\item R-Finance Conference (Chicago, May 2015). ``Matrix Metrics: Network-Based Systemic Risk Modeling."

\item R-Finance Conference (Chicago, May 2015). ``Efficient Rebalancing of Taxable Portfolios."

\item JOIM Meetings (San Diego, April 2015). ``Efficient Rebalancing of Taxable Portfolios."

\item PAN-IIM Meetings  (Google, Mountain View, April 2015). ``Risk Networks''.

\item Moody's Inc.  (San Francisco, February 2015). ``Risk Networks''.

\item HEC (Montreal, November 2014). ``Risk and Return Networks.''

\item QWAFAFEW, Society for Financial Analysts (San Francisco, November 2014). ``Risk and Return Networks.''

\item Seoul National University (Korea, October 2014). ``Risk and Return Networks.''

\item Federal Deposit Insurance Corporation, (Washington DC, July 2014). ``Risk and Return Networks.''

\item International Monetary Fund, (Washington DC, July 2014). ``Risk and Return Networks.''

\item International Risk Management Conference (IRMC) at the Warsaw School of Economics, (Poland, June 2014). ``Risk and Return Networks.''

\item BlackRock Inc., (San Francisco, June 2014). ``Risk and Return Networks.''

\item Santa Clara University, (April 2014). ``Risk and Return Networks.'' 

\item Journal of Investment Management Conference (San Diego, March 2014). ``Social Network Modeling in Finance.'' 

\item California Polytechnic (San Luis Obispo, February 2014). ``Social Network Modeling in Finance.'' 

\item Federal Reserve Bank (San Francisco, January 2014). ``Social Network Modeling in Finance.'' 

\item Villanova University (Philadelphia, November 2013). ``Social Network Modeling in Finance.'' 

\item George Washington University (Washington DC, November 2013). ``Social Network Modeling in Finance.'' 

\item Franklin Templeton Advisors (Daly City, October 2013). ``What is Data Science? Algorithms, Analytics, Applications for Big Data in Finance.'' 

\item Federal Deposit Insurance Corporation conference (Washington DC, October 2013). ``Designed for Failure? Risk-Return Tradeoffs and Risk Management of Structured Investment Vehicles.''

\item Center for Analytic Finance, Indian School of Business (Hyderabad, India, July 2013). ``Credit Spreads with Dynamic Debt.''

\item Center for Analytic Finance, Indian School of Business (Hyderabad, India, July 2013). ``Style Drift: An Analysis of Venture Investing.''

\item Conference on Behavioral Models and Sentiment Analysis applied to Finance (London, July 2013). ``Network Analysis for Systemic Risk and Venture Capital Communities.'' 

\item Conference on Behavioral Models and Sentiment Analysis applied to Finance (London, July 2013). ``Text Mining for Sentiment Analysis.'' 

\item Bank of Korea, Global Initiative Program Seminar (Seoul, June 2013). ``Big Data, Algorithms, and Business Intelligence.''

\item R Finance Conference (Chicago, May 2013). ``R in Academic Finance.''

\item Yahoo! (Santa Clara, April 2013). ``Mathematical Modeling and Data Analytics in Finance.''

\item Georgia State University (Atlanta, April 2013). ``Going for Broke: Optimizing Investments in Distressed Debt'' and ``Credit Spreads with Dynamic Debt.''

\item Hong Kong University of Science and Technology (Hong Kong, March 2013). ``Going for Broke: Optimizing Investments in Distressed Debt'' and ``Credit Spreads with Dynamic Debt.''


\item Midwest Finance Association Meetings (Chicago, March 2013). ``Venture Capital Communities.''

\item BlackRock (San Francisco, January 2013). ``Social Network Modeling in Finance with Text Analytics.''

\item American Finance Association (AFA, San Diego, January 2013). ``Did CDS Trading Improve the Market for Corporate Bonds?''

\item Center for Systemic Risk Analytics (Boston. December 2012). ``Webinar on Data Mining for Systemic Risk Analysis of Bank Colending Networks.''

\item Indian School of Business (Hyderabad, December 2012). 
``Going for Broke: Optimizing Investments in Distressed Debt'' and ``Credit Spreads under Mean-reverting Capital Structure.''

\item EDHEC Risk Institute (Singapore, September 2012). ``Risk, Regulation, and Restructuring of Distressed Mortgage Debt.'' 

\item Singularity University (NASA, Ames, June 2012). ``Risk, Regulation, and Restructuring of Distressed Mortgage Debt.'' 

\item Western Finance Association (WFA) conference (Las Vegas, June 2012). ``Did CDS Trading Improve the Market for Corporate Bonds?''

\item Fourth Stanford Conference in Quantitative Finance (Stanford University, June 2012). ``Restructuring Debt in the Global Economy.''


\item Instititute of Chartered Accountants Conference (Dubai, April 2012). 
``Restructuring Debt in the Global Economy.''

\item Financial Risk Conference, Macquarie University, Center for Financial Risk (Sydney, March 2012).
``Risk Regulations, and Restructuring of Distressed Mortgage Debt.''

\item Macquarie University (Sydney, March 2012).
``Text Analytics and Network Modeling in Finance.''

\item JOIM Conference (San Francisco, March 2012).
``Risk Regulations, and Restructuring of Distressed Mortgage Debt.''

\item R User Group Meeting at Intuit (Mountain View, December 2011). 
``Venture Capital Communities.''

\item R User Group Meeting at Facebook (Palo Alto, November 2011).
``Using R in Academic Finance.''

\item Risk USA Conference (New York, November 2011).
``Did CDS Trading Improve the Market for Corporate Bonds?''

\item Financial Managament Association (FMA, Denver, October 2011).
``Did CDS Trading Improve the Market for Corporate Bonds?''

\item Financial Managament Association (FMA, Denver, October 2011).
``The Principal Principle.''

\item Financial Managament Association (FMA, Denver, October 2011).
``Roundtable on the Future of Housing Finance.''

\item FDIC/JFSR Banking Conference (Washington DC, September 2011). 
``Did CDS Trading Improve the Market for Corporate Bonds?''

\item University of Maryland (Maryland, September 2011). 
``Strategic Loan Modification.''

\item Financial Management Association Applied Conference (New York, May 2011). 
``Did CDS Trading Improve the Market for Corporate Bonds?''

\item National University of Singapore (February 2011). 
``Modeling Recovery.''

\item National University of Singapore (February 2011). 
``The Principal Principle: Optimal Modification of Distressed Home Loans.'' \& ``Strategic Loan Modification: An Options-Based Response to Strategic Default.''

\item Risk USA Workshop (New York, November 2010). 
``Beyond Mean-Variance: Portfolios with Structured Products  and Non-Gaussian Returns.''

\item Risk USA (New York, November 2010). 
``The Principal Principle: Optimal Modification of Distressed Home Loans.'' \& ``Strategic Loan Modification: An Options-Based Response to Strategic Default.''

\item FDIC Conference on Mortgage Finance Reform (Washington DC, October 2010). 
``Strategic Loan Modification: An Options-Based Response to Strategic Default.''

\item IBM Almaden (San Jose, October 2010). 
``The Principal Principle: Optimal Modification of Distressed Home Loans.'' \& ``Strategic Loan Modification: An Options-Based Response to Strategic Default.''

\item Fordham University (New York, October 2010). 
``Strategic Loan Modification: An Options-Based Response to Strategic Default.''

\item Bellatore Investor Offsite (Napa, July 2010).
``The Principal Principle: Optimal Modification of Distressed Home Loans.''

\item QWAFAFEW (San Francisco, April 2010).
``The Principal Principle: Optimal Modification of Distressed Home Loans.''

\item Santa Clara University, Department of Finance (Santa Clara, April 2010).
``The Principal Principle: Optimal Modification of Distressed Home Loans.''

\item Moodys KMV (San Francisco, March 2010).
``The Principal Principle: Optimal Modification of Distressed Home Loans.''

\item Santa Clara University, Department of Mathematics (Santa Clara, February 2010).
``The Principal Principle: Optimal Modification of Distressed Home Loans.''

\item Indian School of Business (Hyderabad, December 2009).
``The Principal Principle: Optimal Modification of Distressed Home Loans.''

\item Risk Conference (New York, October 2009). 
``Recovery Rate Modeling.'' 

\item Risk Conference (New York, October 2009). 
``Modeling Liquidity.'' 

\item Federal Deposit Insurance Corporation (Washington, DC, September 2009).
``Saving Homes and Banks: The Optimal Modification of Distressed Home Loans.''

\item Quant Congress (New York, July 2009).
``Beyond Mean-Variance: Modeling Mental Account Portfolios with Structured Products and Nonlinear Securities.''

\item University of Paris, Dauphine (Paris, June 2009).
``Modeling Recovery Rates.''

\item Claremont McKenna College (Claremont, May 2009).
``Modeling Recovery Rates.''

\item University of Chicago, Liquidity Conference (Chicago, November 2008). 
``Run Lengths and Liquidity''.

\item International Risk Management Conference (Firenze, Italy, June 2008). 
``Accounting-Based versus Market-Based Cross-Sectional Models of CDS Spreads''. 

\item Moody's Credit Conference (New York, May 2008). 
``Hedging Credit: Equity Liquidity Matters.''

\item California Polytechnic (San Luis Obispo 2008). 
``Extracting and Using Credit Information in Hybrid Models''.

\item American Mathematical Society Meetings (San Diego 2008).
``Extracting and Using Credit Information in Hybrid Models'' (in the Special Session 
on Financial Mathematics). 

\item Hong Kong University of Science and Technology (December 2007). 
``Processing Graphical Information: Perceptual Illusions of Risk and Return.''

\item Society for Judgment and Decision Making Conference (Long Beach 2007). 
``Processing Graphical Information: Perceptual Illusions of Risk and Return.''

\item Risk (New York 2007). 
Workshop on ``Innovations in Credit Models''

\item Villanova University (2007). 
``Information Extraction in Hybrid Models'' (presentation of three papers). 

\item Association for Consumer Research Conference (Memphis 2007). 
``Processing Graphical Information: Perceptual Illusions of Risk and Return.''

\item University of Chicago, Stevanovich Center for Financial Mathematics (2007).
``Implied Recovery''. 

\item University of Houston (2007). 
``Information Extraction in Hybrid Models'' (presentation of three papers). 

\item Journal of Investment Management Conference (Boston 2007). 
Discussion of ``Anticipating Correlations'' (by Robert Engle). 

\item Calpers (2007, Sacramento). 
``Run Lengths and Liquidity''.

\item QWAFAFEW (2007, San Francisco). 
``Run Lengths and Liquidity''.

\item Korean Finance Association Conference (2007, Korea). 
``Polishing Diamonds in the Rough: The Sources of Syndicated Venture Performance''. 

\item Moodys KMV (2007, San Francisco).
``Implied Recovery'' 

\item Festschrift for Ed Altman (2006, New York).
``Implied Recovery''.

\item S\&P Credit Congress (2006, New York).
``Implied Recovery''.

\item Journal of Investment Management Conference (2006, Boston). 
``A Simple Model for Pricing  Securities with
Equity, Interest-rate, and Default Risk''.

\item Federal Deposit Insurance Corporation (2006, Washington DC).
``Basel II Technical Issues.''

\item Federal Deposit Insurance Corporation (2006, Washington DC).
``Back to Basics: Fundamentals-Based versus Market-Based Cross-Sectional Models of CDS Spreads''

\item Pacific Investment Management Company (2006, Orange County). 
``Implied Recovery''.

\item Pacific Investment Management Company (2006, Orange County). 
``A Simple Model for Pricing  Securities with
Equity, Interest-rate, and Default Risk''.


\item Barclays Global Investors (2006). 
``Back to Basics: Fundamentals-Based versus Market-Based Cross-Sectional Models of CDS Spreads'' and ``Implied Recovery''. 

\item Barclays Global Investors (2006). 
``Information Structure and Investor Sentiment Extraction using the Internet.''

\item Carnegie-Mellon University (2006). 
``Modeling Correlated Default'' (a presentation of three papers). 

\item Stanford University (2005). 
``Information Structure and Investor Sentiment Extraction using the Internet.''

\item Q-group conference, (Carlsbad 2005). 
``Common Failings: How Corporate Defaults are Correlated''. 


\item Western Finance Association (WFA) conference (Portland, 2005). 
``Common Failings: How Corporate Defaults are Correlated''. 

\item Moody's Credit Conference (London, 2005). 
``Common Failings: How Corporate Defaults are Correlated''. 


\item Derivatives Securities Conference (Washington 2005) 
``A Simple Model for Pricing  Securities with
Equity, Interest-rate, and Default Risk''.


\item Quant Congress (New York, 2004) 
  ``Common Failings: How Corporate Defaults are Correlated''.
  
\item Citigroup (New York, 2004) 
  ``Common Failings: How Corporate Defaults are Correlated''.

\item QFAFAFEW - Quantitative Analysts Society (San Francisco, 2004)
  ``Common Failings: How Corporate Defaults are Correlated''.


\item Federal Deposit Insurance Corporation (Washington, 2004)  
``Common Failings: How Corporate Defaults are Correlated''.

  
\item American Psychology Association (APA) Conference (Honolulu,
  2004) ``Why are Stocks with Shorter Run Lengths Preferred?''.
  
\item Choice Symposium (Colorado, 2004) ``Yahoo for Amazon: Sentiment
  Extraction from Stock Message Boards'' and ``Financial Communities''
  (in the special session on Word of Mouth Effects in Business
  Models).

\item University of Massachusetts (Amherst 2004) 
``A Simple Unified Model for Pricing Derivative Securities with
Equity, Interest-rate, and Default Risk''.

\item Federal Deposit Insurance Corporation (Washington, 2004)  
``A Simple Unified Model for Pricing Derivative Securities with
Equity, Interest-rate, and Default Risk''.


\item American Finance Association Meetings (San Diego, 2004)  
``A Simple Unified Model for Pricing Derivative Securities with
Equity, Interest-rate, and Default Risk''.


\item Lehman Brothers (New York, 2003)
``A Simple Unified Model for Pricing Derivative Securities with
Equity, Interest-rate, and Default Risk''.

\item Morgan Stanley (New York, 2003)
``A Simple Unified Model for Pricing Derivative Securities with
Equity, Interest-rate, and Default Risk''.


\item Credit Risk Summit (New York, 2003)
``A Simple Unified Model for Pricing Derivative Securities with
Equity, Interest-rate, and Default Risk''.


\item European Finance Association Conference (Glasgow 2003).
``Modeling Correlated Default.''

\item Moodys KMV Corporation (San Francisco, 2003)
``A Simple Unified Model for Pricing Derivative Securities with
Equity, Interest-rate, and Default Risk''.

\item Universita La Sapienza (Rome, 2003)
``Modeling Correlated Default.''

\item Universita La Sapienza (Rome, 2003)
``A Simple Unified Model for Pricing Derivative Securities with
Equity, Interest-rate, and Default Risk''.

\item Santa Clara University (Santa Clara, 2003)
``A Simple Unified Model for Pricing Derivative Securities with
Equity, Interest-rate, Default and Liquidity Risk''.

\item Stanford University (Palo Alto, 2003)
``A Simple Unified Model for Pricing Derivative Securities with
Equity, Interest-rate, Default and Liquidity Risk''.

\item University of Oklahoma (Norman, OK 2003)
``A Simple Unified Model for Pricing Derivative Securities with
Equity, Interest-rate, Default and Liquidity Risk''.

\item New York University (New York 2003)
``A Simple Model for Pricing Derivative Securities with
Equity, Interest-rate, Default and Liquidity Risk''.

\item York University (Toronto 2003)
``A Simple Model for Pricing Derivative Securities with
Equity, Interest-rate, Default and Liquidity Risk''.


\item American Finance Association Meetings (Washington 2003)
``Correlated Default Risk''.


\item American Finance Association Meetings (Washington 2003)
``International Portfolio Choice with Systemic Risk.''


\item MSRI Event Risk Conference (New York 2002)
``Correlated Default Risk''.

\item Carnegie-Mellon University (Pittsburgh 2002)
``Correlated Default Risk''.



\item University of Arizona (Tucson 2002)
``Yahoo for Amazon: Sentiment Extraction from Small Talk on the Web''. 

\item Australian Society of Financial Analysts (Sydney 2002)
``Yahoo for Amazon: Sentiment Extraction from Small Talk on the Web''. 

\item Australian Society of Financial Analysts (Sydney 2002)
``The Private Equity Discount: An Empirical Examination of the
Exit of Venture Backed Companies''.


\item Singapore Chapter of Financial Analysts (Singapore 2002)
``Credit Derivatives for Fixed Income Portfolios''.

\item Japan Institute of Securities Practitioners (JISP) (Tokyo, 2002)
``Credit Derivatives for Fixed Income Portfolios''.


\item Barclays Global Investor Seminar (San Francisco, 2002) 
``Modelling Correlated Default''.

\item Risk Conference (Boston, 2002) 
``Modelling Correlated Default''.

\item AIMR Conference (Toronto, 2002) 
``Modelling Correlated Default''.

\item American Finance Association (Atlanta, 2002).
``The Private Equity Discount: An Empirical Examination of the
Exit of Venture Backed Companies''.

\item American Finance Association (Atlanta, 2002).
``e-Information: Preliminary Findings''.

\item Q-Group conference, Phoenix, Arizona (2001). 
``Correlated Default Analysis of Bonds''.

\item University of Wisconsin, Madison (2001).
``Yahoo for Amazon: Opinion Extraction from Small Talk on the Web''. 

\item KMV Corporation, San Francisco (2001).
 ``A Discrete-Time Approach to No-Arbitrage Pricing of Credit
Derivatives with Rating Transitions'' , and  
``Markov Chain Monte Carlo methods for Option Pricing.''

\item European Finance Association Conference, Barcelona, Spain (2001).
``The Private Equity Discount: An Empirical Examination of the
Exit of Venture Backed Companies''.

\item European Finance Association Conference, Barcelona, Spain (2001).
``Yahoo for Amazon: Opinion Extraction from Small Talk on the Web''. 

\item Asia-Pacific Finance Association Conference, Bangkok, Thailand (2001).
``The Private Equity Discount: An Empirical Examination of the
Exit of Venture Backed Companies''.

\item Asia-Pacific Finance Association Conference, Bangkok, Thailand (2001).
``Yahoo for Amazon: Opinion Extraction from Small Talk on the Web''. 

\item London Business School, U.K. (2001). 
``Yahoo for Amazon: Opinion Extraction from Small Talk on the Web''. 

\item Multinational Finance Society Conference, Garda, Italy (2001).
``Yahoo for Amazon: Opinion Extraction from Small Talk on the Web''. 

\item Risk 2001 conference, Boston (2001).
 ``A Discrete-Time Approach to No-Arbitrage Pricing of Credit
Derivatives with Rating Transitions'' , and  
``Markov Chain Monte Carlo methods for Option Pricing.''

\item Western Finance Association Conference, Tuscon, Arizona (2001).
``International Portfolio Choice with Systemic Risk.''

\item Santa Clara University, Santa Clara (2000). 
``Fee Speech: Adverse Selection and the Regulation of Fee 
Structures in Mutual Funds''. 

\item Bombay Stock Exchange, India (2000). Talk on Derivatives Markets.

\item Northwestern University, Chicago (2000). 
``Yahoo for Amazon: Opinion Extraction from Small Talk on the Web''.

\item Price Waterhouse Coopers Risk Institute Conference, New York (2000).
 ``A Discrete-Time Approach to No-Arbitrage Pricing of Credit
Derivatives with Rating Transitions'' , and  
``Markov Chain Monte Carlo methods for Option Pricing.''


\item Northeastern University, Boston (1999). 
``The Regulation of Fee Structures in Mutual Funds: A 
Theoretical Analysis''. 

\item The Fields Institute, Toronto (1999). 
``Pricing Interest Rate Derivatives: A General Approach''. 

\item The Courant Institute of Mathematical Sciences, New York
University (1999). 
``Pricing Interest Rate Derivatives: A General Approach''. 

\item Purdue University, Indiana (1999). 
``Pricing Interest Rate Derivatives: A General Approach''. 

\item Derivatives Securities Conference, Boston (1999).
``Pricing Interest Rate Derivatives: A General Approach''. 

\item RISK99 conference, Boston (1999).
``Pricing Interest Rate Derivatives: A General Approach''. 

\item Western Finance Association Meetings, Santa Monica (1999).
``Fee Speech: Adverse Selection and the Regulation of Fee 
Structures in Mutual Funds''. 

\item Q-conference, Palm Springs (1999). 
``A Case for Theory Driven Experimental Enquiry''. 

\item Money Talks II, India (1999). Special talk on the use
of Real Options for Investment Decisions.

\item University of Antwerp, Belgium (1998). ``Pricing Credit Risk 
Derivatives,'' a special talk for the Risk Management Chair.

\item University of Wisconsin, Madison (1998). 
``The Regulation of Fee Structures in Mutual Funds: A Theoretical Analysis''. 
 

\item Federal Reserve Bank, Kansas City (1998). 
``Pricing Credit Risk Derivatives''.

\item Derivatives Securities Conference, Boston (1998).
``Of Smiles and Smirks: A Term Structure Perspective''. 

\item Stochastic Programming Conference on Asset Liability Management,
Vancouver (1998).
``International Portfolio Choice with Systemic Risk''.

\item Berkeley Program in Finance, Santa-Barbara (1998).
``Of Smiles and Smirks: A Term Structure Perspective''. 

\item University of California, Los Angeles (1997).
``Of Smiles and Smirks: A Term Structure Perspective''. 

\item Citibank, Hong Kong (1997). 
``A Direct Discrete-Time Approach to
Poisson-Gaussian Bond Option Pricing in the Heath-Jarrow-Morton 
Model''. 

\item Erasmus University, Rotterdam (1997). 
``Of Smiles and Smirks: A Term Structure Perspective''. 

\item Stanford University (1997).
``Of Smiles and Smirks: A Term Structure Perspective''. 

\item University of California, Berkeley (1997).
``Of Smiles and Smirks: A Term Structure Perspective''. 

\item Global Derivatives Conference, Paris (1997).
``Pricing Credit Risk Derivatives''.

\item International Association of Financial Engineers Meetings, Boston (1997).
``Of Smiles and Smirks: A Term Structure Perspective''. 

\item Financial Management Association Meetings, Hawaii (1997).
``Of Smiles and Smirks: A Term Structure Perspective''. 

\item Risk Conference, Geneva (1997).
``Pricing Credit Risk Derivatives''.

\item Credit Risk Conference, Erasmus University, Rotterdam (1997).
``Pricing Credit Risk Derivatives''.

\item Asset Pricing Meetings NBER, Boston (1996). 
``International Portfolio Choice with Systemic Risk''.
 
\item Hong Kong University of Science and Technology (1996).
``A Theory of Optimal Timing and Selectivity''.  

\item Virginia Tech, Blacksburg (1996).
``A Theory of Optimal Timing and Selectivity''.  

\item Board of Governors, Federal Reserve, Washington (1996)
``A Direct Discrete-Time Approach to
Poisson-Gaussian Bond Option Pricing in the Heath-Jarrow-Morton 
Model''. 

\item University of Massachusetts, Amherst (1996).
``A Theory of Optimal Timing and Selectivity''.  

\item American Finance Association Meetings, San Francisco (1996).
``The Central Tendency: A Second Factor in Bond Yields''.

\item Western Finance Association Meetings, Sunriver (1996). 
``A Theory of Banking Structure''. 

\item Financial Management Association, New Orleans (1996).
``The Central Tendency: A Second Factor in Bond Yields''.

\item NBER Conference on the Financing of Property and Casualty Risk,
Palm Beach Florida (1996). 
``Comments on 'Pricing Excess-of-Loss Reinsurance Contracts against
Catastrophic Loss', ''. 

\item Econometric Society (Econometrica) conference, New
Delhi (1996). 
``The Central Tendency: A Second Factor in Bond Yields''.

\item Hong Kong University of Science and Technology (1995).
``Pricing Credit Sensitive Debt when Interest Rates, Credit Ratings
and Credit Spreads are Stochastic''.  

\item Santa-Clara University (1995).
``Pricing Credit Sensitive Debt when Interest Rates, Credit Ratings
and Credit Spreads are Stochastic''.  

\item Boston University (1995).
``Pricing Credit Sensitive Debt when Interest Rates, Credit Ratings
and Credit Spreads are Stochastic''.  

\item Citibank N.A., London (1995).
``Pricing Credit Sensitive Debt when Interest Rates, Credit Ratings
and Credit Spreads are Stochastic''.  

\item Federal Reserve Bank-Case Western University, Cleveland (1995).
``Pricing Credit Sensitive Debt when Interest Rates, Credit Ratings
and Credit Spreads are Stochastic''.  

\item National Bureau of Economic Research (NBER), Boston (1995).
``Pricing Credit Sensitive Debt when Interest Rates, Credit Ratings
and Credit Spreads are Stochastic''.  

\item American Finance Association Meetings, Washington, D.C. (1995)
``Jump-Diffusion Processes and the Bond Markets''.

\item Western Finance Association Meetings, Aspen, Colorado (1995).
``Jump-Diffusion Processes and the Bond Markets''.

\item TIMS-ORSA Meetings, New Orleans (1995).
``Pricing Credit Sensitive Debt when Interest Rates, Credit Ratings
and Credit Spreads are Stochastic''.  

\item University of Illinois, Urbana-Champaign (1994).
``Jump-Diffusion Processes and the Bond Markets''.

\item Carnegie-Mellon University (1994).
``Jump-Diffusion Processes and the Bond Markets''.

\item University of Southern California (1994). 
``Jump-Diffusion Processes and the Bond Markets''.

\item University of Pennsylvania, Wharton School of Business (1994).
``Jump-Diffusion Processes and the Bond Markets''.

\item Harvard University, Boston (1994).
``Jump-Diffusion Processes and the Bond Markets''.

\item University of Michigan, Ann Arbor (1994).
``Jump-Diffusion Processes and the Bond Markets''.

\item Cornell University, Ithaca (1994).
``Jump-Diffusion Processes and the Bond Markets''.

\item University of Chicago (1994).
``Jump-Diffusion Processes and the Bond Markets''.

\item Citibank N.A., Hong Kong (1994)
``Jump-Diffusion Processes and the Bond Markets''.

\item Asset Pricing Meetings NBER, Boston (1994)
``Jump-Diffusion Processes and the Bond Markets''.

\item S.P. Jain Institute of Management, India (1994).
Special talk on using derivatives to solve Emerging Markets problems
in India.

\item Eastern Finance Association Meetings, Boston (1994).
``Stochastic Mean Models of the Term Structure''. 

\item European Finance Association Meetings, Brussels (1994).
``How Diversified are Internationally Diversified Portfolios:
Time-Variation in the Covariances between International Returns''.

\item Financial Management Association Meetings, St. Louis (1994).
``Jump-Diffusion Processes and the Bond Markets''.
}

\end{etaremune}








\begin{description}

\item[PROFESSIONAL SERVICE] \mbox{}

\item[Editorial and Advisory Positions] \mbox{}

%\begin{itemize}
Co-Editor, {\it Journal of Derivatives}, 2002--\\
Co-Editor, {\it Journal of Financial Services Research}, 2013--\\
Senior Executive Editor, {\it Journal of Investment Management}, 2003--\\
Associate Editor, {\it Journal of Financial Intermediation}, 2010--\\ 
Scientific Committee, {\it Finance, Ecole de Hautes Etudes Commerciales
  Du Nord}, 2000--.\\
Associate Editor, {\it International Journal of Theoretical and Applied Finance}, 1999-\\
Associate Editor, {\it Review of Derivatives Research}, 1999--\\
Associate Editor, {\it Global Credit Review}, 2012--\\
Advisory Editor, {\it Studies in Economics and Finance}, 2014--\\
Editorial Board, {\it Quantitative Finance Letters}, 2013--2016\\
Editorial Advisory Board, {\it Vikalpa, The Journal for Decision Makers}, IIMA, 2015--\\
Research Advisory Board Member, {\it Association for Investment
Management and Research (AIMR)}, 2000--2014\\
Associate Editor, {\it Financial Analysts Journal}, 2000-2002\\
Associate Editor, {\it Management Science}, 2001-2008\\
Associate Editor, {\it Journal of Financial Services Research}, 1999-2012\\
Associate Editor, {\it The Journal of RISK}, 1999-2005.\\



\item {\bf Refereeing}: Reviewer for the following journals and
organizations:

{\it 
Annals of Operations Research; 
Applied Economics; 
CFA Research Foundation; 
Econometrica; 
European Finance Review; 
European Journal of Finance; 
Finance and Stochastics; 
Financial Analysts Journal; 
Financial Management; 
Financial Practice and Education; 
Finance Research Letters; 
Fuzzy Systems Journal; 
Global Finance Journal; 
Hong Kong Research Council; 
Intelligent Systems in Accounting, Finance and Management; 
International Journal of Theoretical and Applied Finance; 
International Review of Economics and Finance; 
Journal of Banking and Finance; 
Journal of Business; 
Journal of Business and Economic Statistics; 
Journal of Computational Finance; 
Journal of Derivatives; 
Journal of Econometrics; 
Journal of Economics and Business; 
Journal of Economic Dynamics and Control; 
Journal of Empirical Finance; 
Journal of Finance; 
Journal of Financial and Quantitative Analysis; 
Journal of Financial Economics; 
Journal of Financial Engineering; 
Journal of Financial Markets; 
Journal of Financial Research; 
Journal of Financial Services Research; 
Journal of Futures Markets; 
Journal of Investment Management; 
Journal of Marketing Research; 
Journal of Money, Credit and Banking; 
Journal of Risk; 
Journal of Supercomputing; 
Management Science; 
Mathematical Finance; 
National Science Foundation; 
Pacific Basin Finance Journal; 
Quantitative Finance; 
Quarterly Journal of Business and Economics; 
Research Policy; 
Review of Derivatives Research; 
Review of Economics and Statistics; 
Review of Financial Studies; 
Risk; 
Scandinavian Actuarial Journal; 
Spanish Economic Review
}

%\item {\bf Memberships}
%
%American Finance Association\\
%Financial Management Association (program committee 1998, 
%awards committee 2000)\\
%Western Finance Association (program committee, 1997, 1998)\\

\item {\bf Conferences/Programs/Judging}

\begin{etaremune}
\setlength\itemsep{-0.3em}

\item Program Committee, 2016 IEEE Workshop on Big Data for Financial News and Data.

\item Scientific Committee, Moody's Academic Conference December 2016, India. 

\item Academic program organizer for the Journal of Investment Management
Conference: Oxford University, September 2016. 

\item Academic program organizer for the Journal of Investment Management
Conference: Nortwestern University, Chicago, September 2016. 

\item Program Committee for the Data Science for Macro-Modeling (DSMM2016) Workshop in conjunction with SIGMOD 2016 (San Francisco, July 2016).


\item Academic program organizer for the Journal of Investment Management
Conference: Sonoma, March 2016. 

\item Program Committee, Financial Intermediation Research Society (FIRS) Conference 2016. 

\item Scientific Committee, Moody's Academic Conference December 2015, India. 

\item Academic program organizer for the Journal of Investment Management
Conference: Boston, October 2015. 

\item Academic program organizer for the Journal of Investment Management
Conference: San Diego, March 2015. 

\item Program Committee, FDIC/JFSR Banking Conference, Washington DC, September 2014. 

\item Academic program organizer for the Journal of Investment Management
Conference: Napa, September 2014. 

\item Program Committee, Sigmod 2014 Workshop - Data Science for Macro-Modeling: Snowbird, Utah, June 2014.  

\item Academic program organizer for the Journal of Investment Management
Conference: Sam Diego, March 2014. 

\item Academic program organizer for the Journal of Investment Management
Conference: Napa, October 2013. 

\item Judges panel for the S\&P SPIVA awards 2014. 

\item Academic program organizer for the Journal of Investment Management
Conference: Stanford, March 2013. 

\item Academic program organizer for the Journal of Investment Management
Conference: Chicago, September 2012. 

\item Judges panel for the S\&P SPIVA awards 2013. 

\item Academic program organizer for the Journal of Investment Management
Conference: San Francisco, March 2012. 

\item Academic program organizer for the Journal of Investment Management
Conference: Boston, October 2011. 

\item Program Committee, Society for Financial Econometrics, June 2011. 

\item Academic program organizer for the Journal of Investment Management
Conference: San Diego, March 2011. 

\item Academic program organizer for the Journal of Investment Management
Conference: Boston, October 2010. 

\item Program Committee, ``Conference on Risk Management,'' National University of SIngapore, July 2010. 

\item Conference Co-Chair, ``The Value of Values,'' Conference on Social Investing and Corporate Social Responsibility, Santa Clara, May 2010. 

\item Program Committee, Workshop on Parallel and Distributed Computing in Finance, April 2010, Atlanta.; as part of the 24th IEEE International Parallel and Distributed Processing Symposium - IPDPS 2010.

\item Academic program organizer for the Journal of Investment Management
Conference: San Francisco, March 2010.

\item Academic program organizer for the Journal of Investment Management
Conference: Boston, October 2009. 

\item Program Committee, Workshop on Parallel and Distributed Computing in Finance, May 2009, Rome, Italy; as part of the 23rd IEEE International Parallel and Distributed Processing Symposium - IPDPS 2009.

\item Academic program organizer for the Journal of Investment Management Conference: San Francisco, March 2009.

\item Academic program organizer for the Journal of Investment Management Conference: Boston, October 2008. 

\item Academic program organizer for the Journal of Investment Management Conference: New York, April 2008. 

\item Program Committee for the Financial Management Conference, Dallas, Texas, October 2008. Track Chair for Derivatives. 
 
\item Program Committee, The First Workshop on Parallel and Distributed Computing in Finance (Computational Finance), April 2008, Miami; as part of the 22nd IEEE International Parallel and Distributed Processing Symposium - IPDPS 2008.

\item Co-editor of Special Issue of the Journal of Fixed Income (based on the Midwest Finance Conference papers) 2008.

\item Program Committee, Midwest Finance Conference, San Antonio, TX, February 2008. Co-chair of special track on Fixed-Income.

\item Academic program organizer for the Journal of Investment Management Conference: Boston, September 2007. 

\item Academic program organizer for the Journal of Investment Management Conference: San Francisco, March 2007.

\item Academic program organizer for the Journal of Investment Management Conference: Boston, September 2006. 

\item Academic program organizer for the Journal of Investment Management Conference: San Francisco, March 2006; 

\item Conference Organizer for the Conference on ``Event Risk'' (November 2002), organized by the Mathematical Sciences Research Institute, Berkeley, in New York.

\item Organizing committee, Conference on ``Randomized Algorithms in Finance'' March 2001, (Mathematical Sciences Research Institute at Berkeley).

\end{etaremune}

\end{description}


\begin{description}
\item[REFERENCES] \mbox{}

\begin{itemize}
\setlength\itemsep{-0.1em}

\item Professor Darrell Duffie, Stanford University, (duffiedarrell@gsb.stanford.edu) 

%\item Professor Edwin Elton, New York University, (212-998-0361, eelton@stern.nyu.edu)

\item Professor Ravi Jagannathan, Northwestern University, (rjaganna@nwu.edu)

\item Professor Marti Subrahmanyam, New York University, (msubrahm@stern.nyu.edu)

\end{itemize}

\end{description}


%\begin{description}
%\item[EXPERT TESTIMONY] \mbox{}

%2009: Argent Class Convertible Fund LP vs Countrywide Financial Corporation, et al.\\
%2009: In Re Countrywide Financial Corporation Securities Litigation.\\
%2011: Main State Retirement System, et al vs Countrywide Financial Corporation et al. \\

%\end{description}

\end{document}





















